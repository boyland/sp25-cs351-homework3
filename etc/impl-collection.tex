
%\subsection{Interfaces}
%
%A Java ``interface'' is a special kind of ``abstract class.''
%A Java \emph{interface} specifies a set of (public) operations that an ADT
%might be expected to implement.  An ADT implementation signals its
%adherence to the interface by adding ``\texttt{implements $I$}'' 
%to the class header, where
%$I$ is the name of the interface.
%
%A Java variable (field, parameter or local) may have an interface as a
%type.  Of course the variable is \emph{not} an instance of this type,
%since interfaces are abstract---they don't provide any implementation.
%Instead they are instances of some class that implements the interface.
%In Java, it is considered better to use interfaces for the type of
%variables or method returns instead of concrete classes since it makes
%the program more general.

%In this assignment, you will need to use
%an interface called \verb|ActionListener| and
%one ``generic'' interface both described later.

\subsection{Collection ADT}

Soon after Java was released, the need for a standardized collections
framework became apparent.  Accordingly, Java~1.2 introduced a
standard collections framework.  Each of the collection classes
implements a standard set of operations.

For this assignment, you will write a class that implements
the standard Java Collection interface which has the following
methods (in addition to ones every object has):
\begin{description}
    \item[size()] Return the number of elements in the collection.
    \item[isEmpty()] Return whether the collection is empty.
    \item[contains(Object)] Return whether the collection contains the parameter.
    \item[iterator()] Return an iterator over the elements.
    \item[toArray()] Return an array of all elements.
    \item[add($E$)] Add an element to the collection and return true.
    \item[remove(Object)] Remove an element from the collection and
      return true, or return false if it was not found.
    \item[containsAll(Collection)] Return true if every element in the
      parameter is also present in this collection.
    \item[addAll(Collection)] Add all the elements from the parameter
      to this collection, returning true if anything was added.
    \item[removeAll(Collection)] Remove all the elements in the parameter
      from this collection, returning true if anything was removed.
    \item[retainAll(Collection)] Remove all the elements in this
      collection that do not also occur in the parameter; returning
      true if anything was removed.
    \item[clear()] Remove everything from the collection.
\end{description}
\pagebreak
Now, as it happens, some of these operations are more fundamental than
others.  For example, if iterators are working, it's easy to write
\verb|contains|:
\begin{quote}
\begin{verbatim}
public boolean contains(Object o) {
    Iterator<E> e = iterator();
    if (o==null) {
        while (e.hasNext())
            if (e.next()==null)
                return true;
    } else {
        while (e.hasNext())
            if (o.equals(e.next()))
                return true;
    }
    return false;
}
\end{verbatim}
(This code is copyright \copyright 2004 by Sun Microsystems.)
\end{quote}
Notice how the code calls \verb|iterator()| which is the
(unimplemented) \verb|iterator| method in the same class.
As it happens, if iterators are working, then many methods can be
implemented in terms of iterators.

For this reason, the Java collections framework includes
\verb|AbstractCollection|,
an abstract class\footnote{We assume you were taught abstract classes
  in CS~251.}
that does precisely this: implement
everything using iterators, with that crucial part omitted:
the \verb|iterator()| method is defined as ``abstract,'' that is
unimplemented.

Now \verb|size()| is also not implemented, even though it would be
perfectly possible (albeit inefficient) to implement that method with
iterators.  It is left abstract because presumably each collection has
a more efficient way to keep track of the size than iterating through
the whole collection.
The \verb|clear| method is similar: it is easy enough to implement
using iterators:
\begin{quote}
\begin{verbatim}
public void clear() {
    Iterator<E> e = iterator();
    while (e.hasNext()) {
        e.next();
        e.remove();
    }
}
\end{verbatim}
(This code is copyright \copyright 2004 by Sun Microsystems.)
\end{quote}
Indeed the abstract class includes this implementation, but notes that
a more efficient implementation should be possible.

The \verb|add(...)| method is implemented in the
\verb|AbstractCollection| class, but the implementation is not
useful: it simply throws an exception that the operation (add) is
``unsupported.''  Indeed, there's no way one can add an element using
an iterator.  So, again, extenders are encouraged to override this
method with a proper implementation.

%\subsection{Generics}

%A \emph{generic} class has type parameters enclosed in angle brackets,
%for example \verb|class ArrayList<T>|; here $T$ is the only generic
%type parameter.  Inside the body of the class $T$ can be used as a
%regular type.  When the class is used, you need to instantiate it with
%an actual type parameter, for instance \verb|String| or
%\verb|Poset|.  
%There is a problem with writing generic classes containing arrays --
%Java does not allow creating generic arrays directly, \ie, 
%\verb|T[] data = new T[10]| is a compiler error.
%In this assignment, we provide a private helper method
%\verb|makeArray(int)| to create arrays of generic type.

\subsection{Iterators}
An iterator is an object that enables a programmer 
to traverse a container (e.g. a collection) without violating the
data encapsulation principle (\ie, declare data as public).
Java's iterators have the following methods:
\begin{description}
\item[hasNext()] Return true if there still remain elements to be returned.
\item[next()] Return the next element in the container.  If there is
  no such element (in which case \verb|hasNext()| should have returned
  false), then throw an instance of \verb|NoSuchElementException|.
\item[remove()] Remove the last element returned by \verb|next()| from
  the container.  Throws an instance of class \verb|IllegalStateException| if
  \verb|next()| has not yet been called or if the element has already
  been removed.
\end{description}
The \verb|Iterator| interface itself is generic (covered in later assignments) but we will
implement ours to work specifically with \textsf{Particle} objects.
To access all the elements of a gravity simulation, and decide whether to
delete them, one can write:
\begin{quote}
\begin{htmlonly}
\begin{alltt}
\end{htmlonly}
%begin{latexonly}
\begin{program}
%end{latexonly}
for (Iterator<Particle> it = $c$.iterator(); it.hasNext();) \{
   Particle b = it.next();
   if (\w{we don't want element b any more}) it.remove();
\}
%begin{latexonly}
\end{program}
%end{latexonly}
\begin{htmlonly}
\end{alltt}
\end{htmlonly}
\end{quote}
Java has a special syntax of ``enhanced'' for-loops to make it easy to use
iterators.  The shortcut:
\begin{quote}
\begin{htmlonly}
\begin{alltt}
\end{htmlonly}
%begin{latexonly}
\begin{program}
%end{latexonly}
for (Particle b : $c$) \{
   ...
\}
%begin{latexonly}
\end{program}
%end{latexonly}
\begin{htmlonly}
\end{alltt}
\end{htmlonly}
\end{quote}
is short for
\begin{quote}
\begin{htmlonly}
\begin{alltt}
\end{htmlonly}
%begin{latexonly}
\begin{program}
%end{latexonly}
for (Iterator<Particle> _secret = $c$.iterator(); _secret.hasNext();) \{
   final Particle b = _secret.next();
   ...
\}
%begin{latexonly}
\end{program}
%end{latexonly}
\begin{htmlonly}
\end{alltt}
\end{htmlonly}
\end{quote}

\subsection{Fail-Fast Iterators}

When using standard Java collections,
iterators become ``stale'' if the collection changes, except by using
the iterator's own \verb|remove| method.  (This does not happen with
cursors, because the ADT has more control over its cursors.)
It is not legal to
use a stale iterator for anything, even calling \verb|hasNext|.
In other words, if you request an iterator and later add an element to
the collection, then you are not allowed to use the iterator again.
If you want an iterator, you must request a new one.

Implementors of Java's standard 
collections are encouraged to provide \emph{fail-fast}
implementations of iterators which ``usually'' throw an exception
(an instance of \textsf{ConcurrentModificationException})
when a stale iterator is used.

Typically, this ability is handled by adding an integer version stamp to every
collection and iterator.  The version should be incremented inside \emph{every method} that
modifies the collection.  If an iterator notices that the version doesn't
match, it throws the required exception before performing any method.
This is \emph{not} part of the invariant: if the invariant fails, it
is because the ADT implementation contained a bug; if the version stamp doesn't match it
is because the ADT was misused. Therefore, if the versions don't
match, do not check anything in the invariant checker.

The iterator version only changes if the collection was changed under
its control (i.e., using \verb|remove|).

\subsection{Nested classes}

A (non-static)
nested class is interesting: it is considered to be ``inside of'' the
object in which it was created and thus has access to all the
(private) fields and (private) methods as if they were its own.
This is not inheritance: it doesn't actually get any fields or methods
from the surrounding class, but it \emph{can} access them.
It is even more confusing if the nested class extends some other
class.  But we won't do that for this assignment.

Nested classes are often used to implement iterators since an iterator
needs access to private internals of the collection class.
%Nested classes (and anonymous classes) are also used to implement
%listeners, if you don't always want \verb|this| to be its own listener.
Within the iterator, you can use the fields of the outer class directly.
If there are two fields with the same name in the outer and inner classes,
you can refer to the one in the outer class by the syntax 
\emph{OuterClassName.this.outerClassField}.
For example, in this homework, the iterator's invariant checker will
need to call (not assert!) the \verb|_wellFormed| in the outer class
(instead of the one within the iterator), by including the call
\verb|ParticleCollection.this._wellFormed()|.

%\subsection{Variable-Arity Parameters}%
%
%You may have noticed that the driver for Homework~\#3 had a method
%whose parameter had type \verb|HexTile...| and was called with a
%variable number of parameters.  Java supports variable arity methods
%by implicitly creating an array around them and passing the array.
%Thus inside the method body, the variable-arity paranmeters can be
%accessed using normal array operations.

%\subsection{Event Handlers}
%
%With Java's graphics model, events are reacted to, rather than
%detected.  In other words, rather than repeatedly asking ``Did someone
%click my window?'' a Java application program will say ``when someone
%clicks my window, let me know.''  The application program will call 
%\verb|add|$X$\verb|Listener| on the window or applet 
%from which it wants to receive events, where $X$ is the kind of event.
%Then the general event system will
%call a method when the window is clicked.  The object to be notified
%must implement the required interface.  
%
%% One can use anonymous classes as event handlers, but in this homework,
%% we will simply have the applet itself implement the event handling interface And
%% the required methods.
%% In the case of mouse clicking,
%% the interface is named \verb|MouseListener|.  If one is only interested
%% in clicking (as for this assignment), all the required methods can have empty
%% bodies except for the method named \verb|mouseClicked|.
%
%\subsection{Anonymous classes}
%
%This week, we will work with the 
%``action listener'' and ``mouse listener'' interfaces.  The former is simpler:
%\begin{verbatim}
%public interface ActionListener {
%  public void actionPerformed(ActionEvent e);
%}
%\end{verbatim}
%One could define a class that implemented this interface and then
%create an instance of it:
%\begin{verbatim}
%public class MyActionListener implements ActionListener {
%  public void actionPerformed(ActionEvent e) {
%    // do something
%  }
%}
%...
%        new MyActionListener()
%\end{verbatim}
%But frequently, we don't really care what the name of the class is, we only want
%to make an instance of it.  In that case we can create an instance
%of an \emph{anonymous class} using the following syntax:
%\begin{quote}
%\begin{verbatim}
%new ActionListener() {
%  public void actionPerformed(ActionEvent e) {
%    // do something
%  }
%}
%\end{verbatim}
%\end{quote}
%This \emph{looks} as if we are creating an instance of the interface,
%but actually we are creating an instance of a class that is never given
%a (visible) name.  The anonymous class implements the interface and
%defines the method \verb|actionPerformed|.
%
%This syntax can be used to create instances of anonymous subclasses
%of normal classes too.  For instance, the class \texttt{MouseAdapter}
%implements the \texttt{MouseListener} interface with methods that do
%nothing.  One can make an instance that does something different for
%one or more of the methods by using the anonymous class syntax:
%\begin{quote}
%\begin{verbatim}
%this.addMouseListener(new MouseAdapter() {
%    @Override
%    public void mouseClicked(MouseEvent e) { ... }
%});
%\end{verbatim}
%\end{quote}
%%\textbf{NB}: To distinguish between a double-click and a single-click, use
%%\verb|getClickCount| on the mouse event object.  A double click
%%is always preceded by a single click in the same location.
%
%\noindent
%\textbf{NB:} The solution uses anonymous classes for the mouse
%listener and for the action listeners on the terrain buttons.
%
%\begin{latexonly}
%\begin{ignore}
%\section{Concurrent Programming without Race Conditions}
%
%Our animation programs are multi-threaded programs because the
%timer/animation thread is different from the thread that handles
%updating the screen.  Programming for this model requires some careful
%work.  We were sloppy in the last homework, but this homework, you
%will need to be careful to follow these rules:
%\begin{itemize}
%\item If a mutable field can be accessed in multiple threads, it
%  should be marked ``volatile''\footnote{%
%Or be protected by a synchronization lock, but that is too complicated
%for us in CS 351.  CS 552 and 537 have more information on synchronization.}.
%\item You can't change library classes.
%\end{itemize}
%One important consequence of these two simple rules is that a mutable
%collection (instance of library class) must not be accessed in multiple
%threads.  The simple solution is to make sure all shared collections
%are \emph{immutable}: they never change.  
%
%In particular, the
%collection of blocks used in the simulation should never change.
%How then can one add or remove blocks?  There's a trick!  We make a
%\emph{new} collection that is a copy of the list of blocks, make
%changes to it (in one thread) and then install it as the (immutable!) list of
%blocks being simulated.  The field that holds the list is mutable
%(and will be marked ``volatile'').
%
%\subsection{Mutable Data Classes}
%
%If a class consists of fields with setters and getters, we call
%it a ``data class.'' Such a class isn't a classic ADT, but just a way to stick
%things together.  Use getters and setters rather than making the
%fields public (a terrible idea) does have certain advantages.
%In particular they can be overridden by subclasses.  We may see 
%examples of this in the second half of this course.
%A mutable data class doesn't have fixed values and thus it usually best practice
%to \emph{not} override \texttt{equals} or \texttt{hashCode}.
%
%\section{Conversions from String}
%
%In our first week, we implemented \verb|toString()| methods
%for getting a string version of the ADTs.  The reverse methods enables
%ADT values to be written to text files and then read back in.
%All three of our Homework \#~1 ADTs need static \verb|fromString|
%methods;
%we have implemented the most complex one for you: \verb|Time.fromString|.
%We provide test cases for these conversions.
%
%You need to write the following string conversion methods:
%\begin{description}
%\item[Duration\#fromString(String)] (static)
%   Convert a duration string back into a duration.
%\item[Period\#fromString(String)] (static)
%   Convert a period string back into a period.
%\end{description}
%These methods should throw an instance of
%\verb|edu.uwm.cs351.util.FormatException| if an error in the string is
%detected.  In particular, a \verb|NumberFormatException| must be
%caught and converted into a \verb|FormatException| (there is a
%constructor for this purpose).
%\end{ignore}
%\end{latexonly}
