\documentclass[12pt]{article}

\usepackage[problem]{handout}

\pagenumbering{alph}
\plainpagenumbering

\class{CS 351/751: Data Structures \& Algorithms}
\period{SEMESTER}
\title[Quiz \#4]{Name: \hrulefill\hrulefill,\quad \hrulefill @uwm.edu}

\begin{document}
\maketitle

In Homework \#3, we implemented a HexTile collection with the following data structure:
\begin{verbatim}
public class HexTileCollection extends AbstractCollection<HexTile> {
    private HexTile[ ] data;
    private int manyItems;
    private int version;
    
    /** Change the capacity of the dynamic array if needed.
     * @param minimumCapacity the needed capacity
     **/
    private void ensureCapacity(int minimumCapacity) { ... }
...
}
\end{verbatim}
Add the following new method to the class:
\begin{verbatim}
/** Change the size of the collection to match the parameters.
 * If the new size is shorter, elements at the end are lost.
 * If the new size is longer, nulls are added at the end.
 * @param size requested new size
 * @throws IllegalArgumentException if the requested new size is negative
 **/
public void setSize(int size) {
   assert wellFormed() : "invariant broken at start of setSize";
   // TODO
\end{verbatim}
\begin{solution}
\begin{verbatim}
  if (size < 0) throw new IllegalArgumentException("Cannot set size to " + size);
  ensureCapacity(size);
  if (manyItems != size) ++ version;
  for (int i=manyItems; i < size; ++i) {
    data[i] = null;
  } 
  manyItems = size;
  assert wellFormed() : "invariant broken by setSize";
}
\end{verbatim}
\end{solution}

\newpage
\maketitle

In Homework \#3, we implemented a HexTile collection with the following data structure:
\begin{verbatim}
public class HexTileCollection extends AbstractCollection<HexTile> {
    private HexTile[ ] data;
    private int manyItems;
    private int version;
    
    /** Change the capacity of the dynamic array if needed.
     * @param minimumCapacity the needed capacity
     **/
    private void ensureCapacity(int minimumCapacity) { ... }
...
}
\end{verbatim}
Add the following new method to the class:
\begin{verbatim}
/** Add an element to the beginning of the collection.
 * @param element element to add
 **/
public void addFront(HexTile element) {
    assert wellFormed() : "invariant broken at start of addFront";
    // TODO
\end{verbatim}
\begin{solution}
\begin{verbatim}
    ensureCapacity(manyItems+1);
    for (int i=manyItems; i > 0; --i) {
        data[i] = data[i-1];
    }
    data[0] = element;
    ++manyItems;
    ++version;
    assert wellFormed() : "invariant broken at end of addFront";
}
\end{verbatim}
\end{solution}

\newpage
\maketitle


In Homework \#3, we implemented a HexTile collection with the following data structure:
\begin{verbatim}
public class HexTileCollection extends AbstractCollection<HexTile> {
    private HexTile[ ] data;
    private int manyItems;
    private int version;
    
    /** Change the capacity of the dynamic array if needed.
     * @param minimumCapacity the needed capacity
     **/
    private void ensureCapacity(int minimumCapacity) { ... }
...
}
\end{verbatim}
Add the following new method to the class:
\begin{verbatim}
/** Remove and return the last element in the collection.
 * @throws IllegalStateException if the collection is empty
 **/
public HexTile removeLast() {
    assert wellFormed() : "invariant broken at start of removeLast";
    // TODO
\end{verbatim}
\begin{solution}
\begin{verbatim}
    if (manyItems == 0) throw new IllegalArgumentException("cannot remove from empty collection");
    HexTile result = data[--manyItems];
    ++version;
    assert wellFormed() : "invariant broken at end of removeLast";
    return result;
}
\end{verbatim}
\end{solution}

\newpage
\maketitle

In Homework \#3, we implemented a HexTile collection with the following data structure:
\begin{verbatim}
public class HexTileCollection extends AbstractCollection<HexTile> {
    private HexTile[ ] data;
    private int manyItems;
    private int version;
    
    /** Change the capacity of the dynamic array if needed.
     * @param minimumCapacity the needed capacity
     **/
    private void ensureCapacity(int minimumCapacity) { ... }
...
}
\end{verbatim}
Add the following new method to the class:
\begin{verbatim}
/** Replace the entire collection elements with those of the argument
 * If the collection were [t1,t2], and the argument were [t3,t4,t5]
 * then afterwards, this collection is [t3,t4,t5].
 * All iterators are rendered stale if the new collection has a different size
 * @param c collection to replace with, must not be null
 **/
public void replaceAll(Collection<HexTile> c) {
    assert wellFormed() : "invariant broken at start of double";
    // TODO
\end{verbatim}
\begin{solution}
\begin{verbatim}
    ensureCapacity(other.manyItems);
    for (int i=0; i < other.manyItems; ++i) {
        data[i] = other.data[i];
    }
    if (manyItems != other.manyItems) ++version;
    manyItems = other.manyItems;
    assert wellFormed() : "invariant broken by replaceAll";
}
\end{verbatim}
\end{solution}

\end{document}
